
\documentclass{article}
\usepackage[utf8]{inputenc}
\usepackage{tabularx}
\usepackage{amsmath}
\usepackage{array}
\usepackage{graphicx}
%\usepackage{caption}
\usepackage{subcaption}

\title{Kernel (image processing)}
\author{Hrvoje Leventić\\
\emph{hrvoje.leventic@ferit.hr} }
\date{May 2019}

\begin{document}

\maketitle

\begin{abstract}
    In image processing, a \textbf{kernel}, \textbf{convolution matrix}, or \textbf{mask} is a small matrix. It is used for \textit{blurring}, \textit{sharpening}, \textit{embossing}, \textit{edge detection}, and more. This is accomplished by doing a convolution between a kernel and an image. 
\end{abstract}

% Keywords command
\providecommand{\keywords}[1]
{
  \small	
  \textbf{\textit{Keywords---}} #1
}
\keywords{neki, test, ovo, ono}

\tableofcontents

%\begin{twocolumn}

\section{Details}

The general expression of a convolution is 
\begin{equation}
    g(x,y) = \omega * f(x,y) = \sum_{s=-a}^a \sum_{t=-b}^b \omega(s,t)f(x-s,y-t),
\end{equation}
where $g(x,y)$ is the filtered image, $f(x,y)$ is the original image, $\omega$ is the filter kernel. Every element of the filter kernel is considered by $$-a \leq s \leq a$$ and $-b \leq t \leq b$.

Depending on the element values, a kernel can cause a wide range of effects. The above are just a few examples of effects achievable by convolving kernels and images. 

\begin{figure}
    \centering
    \begin{subfigure}[t]{.45\textwidth}
        \centering\includegraphics[width=\textwidth]{Vd-Unsharp_5x5.png}
        \caption{pero}\label{fig:pero}
    \end{subfigure}
    \begin{subfigure}[t]{.45\textwidth}
        \centering\vbox{%
        $$
            \begin{bmatrix}
             1 & 1 & 1 \\
             1 & 1 & 1 \\
             1 & 1 & 1 \\
            \end{bmatrix}
        $$
        }
        
        \caption{pero}\label{fig:pero}
    \end{subfigure}
    \caption{Caption}
    \label{fig:my_label}
\end{figure}

\begin{itemize}
    \item Prvi dio
    \item drugi dio
    \item treći dio
    \item alo alo 
\end{itemize}

\subsection{Origin}

\begin{figure}
    \centering
    \begin{subfigure}[t]{.45\linewidth}
        \centering\includegraphics[width=\textwidth]{Vd-Blur1.png}
        \caption{Prvi subfigure}\label{fig:1a}
    \end{subfigure} \qquad
    \begin{subfigure}[t]{0.45\linewidth}
        \centering\includegraphics[width=\textwidth]{Vd-Blur1.png}
        \caption{Prvi subfigure lorem ipsum dolor sit amet abrakadabra alo loreme alo ipsume alo braleee!}\label{fig:1b}
    \end{subfigure}
    \caption{Caption}
    \label{fig:my_label}
\end{figure}

Na koji način ide referenciranje? Referenciramo korištenjem labela koje smo zadali. Na slici \ref{fig:1a} je prvi subfigure, a na slici \ref{fig:1b} je drugi subfigure, dok u konačnici možete cijelu sliku referencirati pomoću \ref{fig:my_label}.
\section{Convolution}

\subsection{Edge Handling}

\subsection{Normalization}

\section{References}

\section{External Links}

%\end{twocolumn}

\end{document}
